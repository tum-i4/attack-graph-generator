\subsection{Vulnerability Scanners}
\label{chap:vulnerability_scanners}

 A vulnerability is a system weakness that can be exploited by a malicious actor with the help of an appropriate suite of tools. Many vulnerabilities are publicly known (such as those in the Common Vulnerabilities and Exposures (CVE) list) and organized in databases, such as the National Vulnerability Database (NVD). CVE\footnote{\url{https://cve.mitre.org/}} is a list of publicly known cybersecurity vulnerabilities where each entry contains an identification number, a description, and at least one public reference. This list of publicly known vulnerabilities is organized in the  NVD\footnote{\url{https://nvd.nist.gov/}} repository, which enables automation of vulnerability management, security measurement, and compliance \cite{booth2013national}. Vulnerability scanners attempt to detect weaknesses by scanning a single host and generating a list of exploitable vulnerabilities \cite{deraison1999nessus, farmer1990cops}. However, more sophisticated approaches are required because many attacks are network-based and performed in multiple steps throughout a network. Therefore, combinations of vulnerability scanners and topologies are considered promising solutions to this problem\cite{sheyner2002automated, ingols2006practical}.
