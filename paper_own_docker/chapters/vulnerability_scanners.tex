\subsection{Vulnerability Scanners}
\label{chap:vulnerability_scanners}

The rise in the usage of microservices and the frequent service communication makes it crucial for data to be transferred and stored securely, while at the same time minimizing vulnerabilities that can hinder normal system operation. A vulnerability is a system weakness that could be exploited by a malicious actor with the help of an appropriate suite of tools. Many vulnerabilities are publicly known (CVE) and organized in databases (NVD). CVE\footnote{\url{https://cve.mitre.org/}} is a list of publicly known cybersecurity vulnerabilities where each entry contains an identification number, a description, and at least one public reference. This list of publicly know cybersecurity vulnerabilities is organized in the  NVD\footnote{\url{https://nvd.nist.gov/}} repository that enables automation of vulnerability management, security measurement, and compliance \cite{booth2013national}. Vulnerability scanners try to detect weaknesses by scanning a single host and generating a list of exploitable vulnerabilities \cite{deraison1999nessus, farmer1990cops, clair}. However, since many attacks are network-based and performed in multiple steps through a network, more sophisticated approaches are required. Therefore a combination of vulnerability scanner and topology is seen as a promising solution to this problem in previous work \cite{sheyner2002automated, ingols2006practical}.
