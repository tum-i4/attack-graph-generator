\subsection{Attack Graphs}
\label{chap:attack_graphs}

Attack graphs \cite{sheyner2002automated} are a popular way to examine network security weaknesses. They facilitate careful analysis of a given system and detection its vulnerable components. The definition of an attack graph may vary, however, it is essentially a directed graph comprising nodes and edges with various representations.

Seyner et al. defined an attack graph as a tuple of states, transitions between the states, an initial state and success states. An initial state represents the state from which the attacker begins an attack and through a chain of atomic attacks attempts to reach one of the success states \cite{sheyner2002automated}. Ou et al. introduced the notion of a logical attack graph, which is a bipartite directed graph comprising fact and derivation nodes. Each fact node is labeled with a logical statement in the form of a predicate applied to its arguments, while each derivation node is labeled with an interaction rule used in the derivation step. The edges in a logical attack graph represent a "depends on" relation \cite{ou2006scalable}. Ingols et al. made a distinction between full, predictive, and multiple-prerequisite (MP) attack graphs. A full graph is a directed acyclic graph comprising nodes that represent hosts and edges that represent vulnerability instances. Predictive attack graphs use the same representation as full attack graphs, with the only difference lying in the constraint of when the edges are added to the attack graph. Note that predictive graphs are generally smaller than full graphs. An MP attack is an attack graph with contentless edges, state nodes, vulnerability instance nodes, and prerequisite nodes \cite{ingols2006practical}.

In this paper, we define an attack graph as a directed acyclic graph with a set of nodes and edges similar to the full graph representation proposed by Ingols et al. \cite{ingols2006practical}. As an expansion to this model, a node represents the state of a host with its current privilege, and an edge represents a successful transition between two such hosts. We can consider an edge as a successful vulnerability exploitation initiated from a host with a required privilege to another or the same host with a newly gained privilege as a result of the vulnerability exploitation. To the best of our knowledge, attack graphs have been used for networks but not microservices, potentially because there is currently no existing tool support. 