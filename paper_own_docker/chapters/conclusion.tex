\section{CONCLUSIONS AND FUTURE WORK}
\label{chap:conclusion}

Microservices are a promising architectural style that advocate practitioners to build systems as a group of small connected  services. Although this style enables better scalability and faster deployment, the full container-based automation within this style raises many security concerns. In this paper, we proposed to use automated attack graph generation as part of the practices of developing microservice-based architectures. Attack graphs aid the developers in identifying attack paths that consist of multiple vulnerability exploitation in the deployed services. The manual construction of attack graphs is an error-prone, resource consuming activity. Hence, automating this process does not only guarantee efficient construction but also complies with the spirit of DevOps practices. We have shown that such automation, extending previous works in computer networks field, is efficient and scales to complex and big microservice-based systems. 

As a future work, we plan to extend this work to support more frameworks that are used in microservices systems. We also plan to study the possible analysis of the resulting attack graphs for purposes of attack detection, and post-postmortem forensics investigations. 