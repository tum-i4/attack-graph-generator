\section{CONCLUSIONS AND FUTURE WORK}
\label{chap:conclusion}

Microservices are a promising architectural style that encourage practitioners to build systems as a group of small connected  services. Although such architectures can realize better scalability and faster deployment, full container-based automation raises many security concerns. In this paper, we have proposed the use of automated attack graph generation relative to the development of microservice-based architectures. Attack graphs help developers identify attack paths that comprise exploitable vulnerabilities in deployed services. Manual construction of attack graphs is an error-prone, resource consuming activity; therefore, automating this process guarantees efficient construction and complies with the spirit of DevOps practices. By extending previous work in field of computer networks, we have demonstrated that such automation is efficient and scales to large and complex microservice-based systems. 

In future work, we plan to extend this work to support more frameworks used in microservice systems. We also plan to study possible analysis of the resulting attack graphs for use in attack detection and post-postmortem forensics investigations. 