\subsection{Microservices}
\label{chap:microservices}

As real-world software increases in size, there is an increasing need to decompose software into an organized structure to promote scalability, reusability, and readability. A software application with modules that cannot be executed independently is referred to as a monolith. Monolithic systems  are characterized by tight coupling, vertical scaling and strong dependence \cite{microservicesfrowler}. The Service Oriented Architecture (SOA) addresses these issues by restructuring its elements into components that provide services that are used by other entities via a networking protocol \cite{papazoglou2003service}. However, in a typical SOA, the services are monolithic which gives rise to the concept of microservices  to provide even more fine-grained task separation \cite{ahmadvand2016requirements}. The term "microservices" was first introduced in 2011 at an architectural workshop as a common term to describe the work of multiple researchers \cite{dragoni2017microservices, microservicesfrowler}. In the microservices paradigm, multiple services are split into very basic task-oriented units. According to Dragoni et al., a microservice is a cohesive, independent process interacting via messages. These microservices constitute a distributed architecture called a microservice architecture \cite{dragoni2017microservices}. Microservice architectures have more heterogeneous technologies, cheaper scaling, resilience, organizational alignment, and composability \cite{newman2015building}. However, they add additional complexity and have a wider attack surface as the need for many services to communicate with each other and third-party software increases \cite{combe2016docker, dragoni2017microservices}. While microservices are an architectural principle, container technology has emerged in cloud computing to provide a lightweight virtualization mechanism. Container technology enables microservices to be packaged and orchestrated through the Cloud \cite{pahl2016microservices}.

