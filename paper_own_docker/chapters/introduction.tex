\section{INTRODUCTION}

Attack graphs are a popular way of examining security aspects of network. They help security analysts to carefully analyze a system connection and detect the most vulnerable parts of the system. An attack graph depicts the actions that an attacker uses in order to reach his goal.  

Statistics. 
https://banyanops.com/blog/analyzing-docker-hub/
https://blog.acolyer.org/2017/04/03/a-study-of-security-vulnerabilities-on-docker-hub/
http://dance.csc.ncsu.edu/papers/codaspy17.pdf \cite{shu2017study}

They do not offer any performance comparison between different topologies. 

In this work, we first get familiar with the basic attack graph and microservice terminology in Section \ref{chap:background}, then present the architecture of our system in Section \ref{chap:method}, perform evaluation in Section \ref{chap:eval} and at the end present what others did in the area in section \ref{chap:related_work}, a conclusion in Section \ref{chap:conclusion} and future work directions in Section \ref{chap:future_work}.

